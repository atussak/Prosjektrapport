\chapter{\abstractname}

%TODO: Abstract

%A number of different principles for snake robot locomotion have been proposed and tested, many of which are based on heuristic rules and stiff position controlled joints. A more physics-based and compliant method is being developed which is based on the formalism of hybrid position/force control. In this assignment you will study this emerging technique and simulate an idealized two-dimensional snake robot propelled by contact forces based on hybrid position/force control in an idealized environment comprising frictionless movement and extentionless environmental assets (“obstacles”).\\


Snake robots are robust serial link robots able to propel themselves through uneven and irregular terrain. As opposed to traditional mobile robots, the snake robot can utilize obstacles found in the environment to push itself forward along a predefined path. The aforementioned mode of locomotion is referred to as \textbf{obstacle aided locomotion (OAL)}. The challenge of both controlling the shape and contact forces of the snake robot has motivated the study of \textbf{hybrid position/force control (HPFC)}, with special emphasis on application within OAL.

%\textbf{Obstacle aided locomotion (OAL)} is a mode of snake robot locomotion in which obstacles are utilized for propulsion, rather than avoided. 



The main principle of HPFC together with a simplified 2-dimensional model of the snake robot and its environment are the backbone of the developed MATLAB simulator. The objective of this simulator is to visualize the behavior of the snake robot and OAL in simple path following scenarios.

%\textbf{Hybrid position/force control (HPFC)} is studied with emphasis on application within OAL and general simultaneous contact force- and shape control of snake robots. The main principle of HPFC together with a simplified model of the snake robot and its environment are the backbone of the developed MATLAB simulator. The objective of this simulator is visualization of the snake robots behavior and OAL in simple path following cases.

Relevant experiments conducted during this study has proven the simulator to be a great resource for presenting concepts and study limitations and possibilities within OAL. However, a geometrical approximation was made to the applied part of the HPFC concept during development of the simulator, which has led to some deviations from the true dynamics of the system and the laws of conservation of energy and momentum to be violated. This is further discussed in the report.

The code of the simulation program can be provided upon request.


\makeatletter
\ifthenelse{\pdf@strcmp{\languagename}{english}=0}


\makeatother



%TODO: Abstract in other language




