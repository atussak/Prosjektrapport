\chapter{Conclusion}\label{outlook}

The project report has presented a MATLAB simulator developed to visualize the concepts of HPFC, OAL and the combination of the two, also known as HOAL. It has been a valuable resource for gaining more insight into the theory and existing ideas, as well as shed light on what challenges remain to be solved.

From the experiments, OAL is concluded to be a simple and intuitive concept with a lot of potential. This conclusion is drawn based on the experiments of the snake robot propelling forward.
Despite the approximations made, the fraction of the HPFC idea applied to the simulator has proven that the inclusion of HPFC in realising the concept of HOAL is a step in the right direction. This is however not based on any direct use of HPFC, but rather the lack of it. More particularly, the fraction applied cannot be considered hybrid \textit{position/force} control, as it is more of a shape control adapted to the constraints through the projections from the HPFC theory of West and Asada \cite{west1985method}.
Trying to foresee the consecutive forces resulting from the movement and shape of the robot and thereby finding an optimal configuration of path and obstacles is a cumbersome operation. It is also a poor and very simplified way of controlling the propulsion, and thus it is obvious that explicit control of the forces should be included into this operation and that HPFC is a significant idea.

Consequently, it has been proven that the constraints, which are the foundation of the spaces in which shape and force can be controlled, are a significant part of the problem. In this project, the model of the robot dynamics excluded these constraint and instead introduced them in an approximation of the velocities, which again led to the laws of conservation of energy and momentum being violated. It is assumed that a more integrated and implicit definition of these constraints would enable the OAL method to benefit further from HPFC.

%However, the constraints should be integrated and implicitly defined in the dynamics of the robot to benefit further from HPFC.

Finally, it can be stated that the concept of HOAL has been verified by means of analyzing visual experiments, but needs to be researched further for a more mathematical and solid basis. Future work regarding the challenges around this is presented below.

\section{Future work}

The most significant challenge per now is concluded to be defining and analyzing the constraints and resulting constraint forces between the snake robot and the environment. As the method of West and Asada \cite{west1985method} is not considering the dynamics of the system, it is proposed to adapt the method of Yoshikawa \cite{yoshikawa1987dynamic} to the snake robot in future work. This would yield a dynamical model with the obstacle constraints integrated, rather than introducing them explicitly and adding the resulting forces subsequently.

When it comes to the path following performance, a clear deficiency in the applied method is the calculation of the desired angles. As mentioned in Chapter \ref{ch:discussion}, solving this challenge by means of an optimization problem would most probably return more satisfying results.

All experiments and calculations are based on the knowledge of the desired path. The path and obstacles are thus manually designed to achieve the presented results. However, a person will not necessarily see the most optimal path in a cluttered environment, nor is it a realistic action to define it manually in real life scenarios. 
%If the decomposition of the snake robot task space explained in \ref{subseq:snakeHPFC} is clearly and uniquely defined, the knowledge of these spaces could be exploited in finding a path that .
Methods like model predictive control (MPC) or reinforcement learning (RL) are proposed for future work in finding the optimal path as the robot moves through the environment.

%Finally, knowledge about the specified spaces may be exploited in planning of a path from one point to another. In particular, it can be used to design an optimal path that maximizes the propulsion space. When the corresponding optimal forces are known, HPFC can be used to realise these forces while adjusting the shape of the robot to the path.