\chapter{Discussion} \label{ch:discussion}

\section{Effect of simplifications}

The geometrical approximation made regarding the contact forces between the robot and the obstacles have been characteristic for the outcome of the simulations. Firstly, it might not yield unique solutions, leading to unpredictable or unrealistic behavior in some scenarios. Secondly, it has been observed that the robot in some cases ends up moving slightly over the edge of obstacles when there are several "conflicting" constraints present.

There are of course several methods that could have avoided this simplification.
One option would be to implicitly define the forces as a part of the dynamical model, in the same way as the forces between the joints in the robot are implicitly defined. The method of Yoshikawa \cite{yoshikawa1987dynamic}, explained in \ref{subseq:dynhpfc}, follows this approach. It does however only consider constraints on the end effector of the robot, and not on arbitrary links.

\hl{Could we have used Paf to find the component belonging to the constraint++ space???}

Another consequence of the geometrical approximation, is that the projection matrices only allow movement along obstacles. This is necessary for preventing the robot in moving through obstacles, but also prevents the robot from perpendicularly moving away from them. In other words, whenever a link comes in contact with an obstacle, it will stick to it until it has slid along it. This strict positioning of the links can come in conflict with the controller and lead to behaviors like in \ref{subseq:case23}, where the rear part of the robot was completely off track when leaving the obstacles. The only case in which a link "detaches" from the obstacle before sliding all the way along it, is when it in a discrete time step is projected to a position where it no longer is considered in contact and fazed by the obstacle. 

A workaround for the rigid definition of contact, in which the robot is either completely in contact with the obstacle or not at all, could be introducing an elastic radius or force field around the obstacles, and thus damp the nonlinear effects of the interactions. %PD controller


\section{Insights from experiments}

On the more constructive side, it is clear that the simulator has proven to be a great resource for presenting concepts and study the possibilities within obstacle aided locomotion. Furthermore, the modular architecture of the program, where controller, dynamics, path following etc. is decoupled, allows for it to be effortlessly modified.

The experiments have shown that the positioning of the obstacles and path in relation to each other is vital for the propulsion of the robot. This is especially the case in an environment where the possibility to aid friction for propulsion is absent. Furthermore, it has been shown that it is necessary to have a sufficient number of obstacles. Not only for the propulsion, but also for continuously aiding the robot with alignment along the path. Consequently, a limitation of the implemented system is that the configuration of obstacles and desired path need to be determined manually.

The method of finding the desired angles based on projection onto the path is not robust in cases where the links are far from the path (see experiment in \ref{subseq:case12}). The simplest solution would be avoiding the projection of links that are very misplaced. A future, more advanced, solution would be redefining the optimal path to overlap with the current position of the robot.

A further observation from the experiments, in particular case 2.3 in \ref{subseq:case23}, is that the obstacles used for aligning the rear part of the robot work in a similar fashion as a manipulator base. When the robot has little room for movement in perpendicular directions, its control over the proceeding links is greater because it can resist the Coriolis effects osv eeeh. \hl{mhm?} This does in turn lead to the robot getting further with more links, as the manipulator base analogy is in place for a longer period of time.

When it comes to the computational performance of the simulator, it is observed that the program requires significantly more time for computing the initialization of robots with 6 links or more than that of robots with fewer links. \hl{Skrive hvor mye tid?} An improvement would have been defining the equations of motion directly rather than performing symbolic math differentiation to derive them. However, the initialization only has to run once for every configuration, and the real time performance of the actual simulation is still satisfactory for sample times greater than 0.0001 seconds.

\section{HPFC in snake robots}

Limitations of HPFC in snake robots.



