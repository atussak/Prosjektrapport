\chapter{Sammendrag}

%Snake robots are robust serial link robots able to propel uneven and irregular terrain. As opposed to traditional mobile robots, the snake robot can utilize the obstacles found in the environment to push itself forward along a predefined path. The aforementioned mode of locomotion is referred to as \textbf{obstacle aided locomotion (OAL)}. The challenge of both controlling the shape and contact forces of the snake robot has motivated the study of \textbf{hybrid position/force control (HPFC)}, with special emphasis on application within OAL.

%The main principle of HPFC together with a simplified 2-dimensional model of the snake robot and its environment are the backbone of the developed MATLAB simulator. The objective of this simulator is to visualize the snake robots behavior and OAL in simple path following scenarios.

%From relevant experiments, the simulator has proven to be a great resource for presenting concepts and study limitations and possibilities within OAL. However, a geometrical approximation was made to the applied part of the HPFC concept during development of the simulator, which has led to some deviations from the true dynamics of the system. The code of the simulation program can be provided upon request.

Slangeroboter er robuste seriekoblede roboter som evner å traversere ujevne og uregelmessige terreng. I motsetning til tradisjonelle mobile roboter, kan slangeroboter utnytte hindringer i miljøet sitt ved å dytte seg selv fram langs en forhåndsdefinert bane. Denne bevegelsesarten blir omtalt som \textbf{obstacle aided locomotion (OAL)} eller hindringsbasert fremdrift. Utfordringen som trer fram ved kontrollering av både formen og kontaktkreftene til slangeroboten har motivert studiet av  \textbf{hybrid position/force control (HPFC)} eller hybrid posisjons- og kraftstyring, med spesiell vekt på applikasjoner innenfor OAL.

Hovedprinsippet bak HPFC kombinert med en todimensjonal modell av slangen og dens miljø er selve fundamentet til den utviklede MATLAB-simulatoren. Formålet med denne simulatoren er å visualisere slangerobotens oppførsel og OAL i enkle banefølgingsscenarioer.

Simulatoren har fra relevante eksperimenter utført under studiet vist seg å være et godt verktøy for presentering av konsepter og studere både begrensninger og muligheter knyttet til OAL. På en annen side har en geometrisk approksimasjon av den anvendte delen av HPFC i simulatoren ført til noen avvik fra den sanne dynamikken i systemet og brudd med lovene om konservasjon av energi og bevegelsesmenge. Dette er ytterligere diskutert i rapporten.

Koden for simuleringsprogrammet er til disposisjon ved ønske.

\makeatletter
\ifthenelse{\pdf@strcmp{\languagename}{english}=0}


\makeatother



%TODO: Abstract in other language




