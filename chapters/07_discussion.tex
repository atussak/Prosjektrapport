\chapter{Discussion} \label{ch:discussion}

This chapter looks at the effects of simplifications and approximations made during the development of the simulator, both in regard to the dynamical model and in regard to the path following method. These effects, as well as the general results from the experiments are discussed and some improvements are proposed.

\section{Effects of the contact behaviour simplification}

The geometrical approximation made regarding the contact forces between the robot and the obstacles have been characteristic for the outcome of the simulations.
%Firstly, the approximated method does not yield unique solutions at all times, leading to unpredictable or unrealistic behavior in some scenarios.
From a physical perspective, it is obvious that projecting the velocities to an "allowable space" just based on the positioning of the obstacles differs from actual velocities after a collision. Simply projecting the velocities means that the velocity of the robot body will in some cases increase, and sometimes decrease, after contact with an obstacle. This does in turn mean altering the total energy of the system, which contradicts basic laws of physics under the assumption of frictionless contact. In the experiment in \ref{subseq:case3}, the energy of the robot was almost tripled after contact with a single obstacle.

Additionally, it was observed that the momentum was not conserved through the collision, neither in the x nor y-direction.
Although these observations do not make out a general quantification of the error, it is enough for recognizing that the modelling of interactions in the simulator are considerably off. A further consequence is also that the robot can inherit abrupt and unrealistic movements, but because the dynamics of the employed model have been very slow in all experiments, it has not been a dominating factor and the controller has been able to damp all fast motions. Lastly, it has to be mentioned that the velocity projections can in some cases contribute to additional propulsion of the robot as they introduce new energy to the system.

%The solution to how the robot reacts to a collision is thus not unique or predictable, leading to a somewhat unrealistic behavior.

There are of course several methods that could avoid this simplification.
One option is to implicitly define the forces as a part of the dynamical model, just like the forces between the joints in the robot are implicitly defined. The method of Yoshikawa \cite{yoshikawa1987dynamic}, explained in \ref{subseq:dynhpfc}, follows this approach. It does however only consider constraints on the end effector of the robot, and not on arbitrary links. The adjustment that has to be made in the snake robot case, is that the constraint hypersurfaces and forces onto these surfaces have to be defined for every contact point.

%\hl{Could we have used Paf to find the component belonging to the constraint++ space???}

Another consequence of the geometrical approximation, is that the projection matrices only allow movement along obstacles. This is necessary for preventing the robot from moving through obstacles, but also prevents the robot from perpendicularly moving away from them. In other words, whenever a link comes in contact with an obstacle, it will stick to it until it has slid all the way along it. This strict positioning of the links can come in conflict with the controller and path projection. The only case in which a link "detaches" from the obstacle before sliding along it, is when it in a discrete time step is projected to a position where it no longer is considered in contact and fazed by the obstacle.

Yet another important remark, is that the program treats the cases in which a link is on the edge of the obstacle radius and within the obstacle radius equally. In particular, it considers both cases a point contact and disregards how close it is to the actual obstacle point. In \ref{subseq:case23} it is pointed out that the robot slides slightly through this radius, but it should be kept in mind that for the robot it is the same situation as sliding on the edge of the obstacle. However, it is still a fault that the robot ended up within the radius in the first place, but is most probably a result of the discrete nature of the simulator.

This discrete behavior is also the sole purpose of the obstacle radius, preventing links from jumping over obstacles in discrete time steps, and has in this regard proven to be a successful mechanism.
Furthermore, it is the velocity rather than position of the robot that is changed with the projection. The robot might thus still be inside the obstacle radius at the next time step and the velocity is projected over again. This can in turn lead to inconsistent velocity projections and a twitching behaviour as seen in \ref{subseq:case11}.
Decreasing the radius would naturally also decrease this effect, but it is still crucial to keep the radius in discrete-time simulations. The effect can also be decreased with a smaller sample time, but likewise this is no permanent solution to the problem.

A workaround for the rigid definition of contact in which the robot is either completely in contact with the obstacle or not at all, could be introducing an elastic radius or force field around the obstacles, and thus damp the nonlinear effects of the interactions. %PD controller


\section{Review of the path alignment method}\label{sec:pathdiscussion}

The method of finding the desired angles based on projection onto the path is not robust in cases where the links are far from the path (see experiment in \ref{subseq:case12}). The simplest solution to this would be avoiding the projection of links that are very misplaced. A future, more advanced, solution would be redesigning the optimal path to overlap with the current position of the robot. In \ref{subseq:case12} it is pointed out that the method of finding the desired joint angles disregards the positions of the two first joints, or rather the position of the start- and end point of the tail link. Correcting this poor quality would improve the performance of the path following capability significantly. Situations like in \ref{subseq:case24}, where the tail of the robot gets stuck, would then be avoided.

The desired joint angles from the path projection method are calculated separately for each link in the current program. A more robust and intelligent solution would be defining an objective function that considers the deviation from the path for all joints, and then find the set of desired joint angles minimizing the complete set of deviations. By this method it would also be possible to weight the error of the snake robot head the most. Holden \cite{holden2014optimal} has formulated an optimization problem that finds input torques by minimizing energy consumption while achieving propulsion along the desired path. The method is a great inspiration, but not quite yet a solution to the problem as it still requires the desired joint angles at the obstacles to be known.

\section{Further insights from experiments}

On the more constructive side, it is clear that the simulator has proven to be a great resource for presenting concepts and study the possibilities within obstacle aided locomotion. Furthermore, the modular architecture of the program, where controller, dynamics, path following etc. is decoupled, allows for it to be effortlessly modified.

The experiments have shown that the positioning of the obstacles and path in relation to each other is vital for the propulsion of the robot. This is especially the case in an environment where the possibility to aid friction for propulsion is absent. Furthermore, it has been shown that it is necessary to have a sufficient number of obstacles. Not only for the propulsion, but also for continuously aiding the robot with alignment along the path. Consequently, a limitation of the implemented system is that the configuration of obstacles and desired path need to be determined manually.

A further observation from the experiments, in particular \ref{subseq:case23}, is that the obstacles used for aligning the rear part of the robot are comparable to a manipulator base. More specifically, the robot is in this position able to keep its rear links fixed in the perpendicular direction by pushing against these obstacles. Thus, the control of the proceeding links is increased in the perpendicular direction.
This does in turn lead to the robot being able to get further with a greater number of links, as the robot slides through and stays in touch with the aligning obstacles for a longer period of time.

When it comes to the computational performance of the simulator, it is observed that the program requires significantly more time for computing the initialization of robots with 6 links or more than that of robots with fewer links. An improvement would be defining the equations of motion directly rather than performing symbolic math differentiation to derive them. However, the initialization only has to run once for every configuration, and the real time visual performance of the actual simulation is still satisfactory for sample times greater than 0.1 ms.



