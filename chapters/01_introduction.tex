% !TeX root = ../main.tex
% Add the above to each Chapter to make compiling the PDF easier in some editors.

\chapter{Introduction}\label{Chapter:introduction}

\section{Previous work}
The department of engineering cybernetics at the Norwegian University of Science and Technology (NTNU) has made significant contributions to the field of snake robot control, related to both aquatic snake-like propulsion and efficient snake robot locomotion on flat surfaces \cite{StavdahlNote}.

However, snakes can also utilize irregularities in a terrain. Consequently, Transeth et al. \cite{transeth2008snake} suggested the term \textit{Obstacle Aided Locomotion (OAL)} for snake robots that actively use walls or external objects for means of propulsion.

A relevant concept for attainment of this type of propulsion is \textit{Hybrid Position/Force Control (HPFC)}. The concept was first introduced by Raibert and Craig et al. in 1981 \cite{raibert1981hybrid}. West and Asada \cite{west1985method} further proposed a method for the design of HPFC controllers in 1985 for constrained manipulators that are in contact with the environment at multiple points. This method aims at controlling the position and force at the manipulator joints with kinematics-based projections such that the motion at the end effector, and the force at the contacts with the environment are those required for performing the task. Yoshikawa \cite{yoshikawa1987dynamic} advanced the constraint analysis by taking the dynamics into consideration.


Stavdahl \cite{StavdahlNote} proposed the combination of OAL and HPFC, leading to the term \textit{Hybrid Obstacle Aided Locomotion (HOAL)}. It is still a small area of research, but has a lot of potential in the area of e.g. rescue robots in cluttered environments. Klafstad \cite{TorjusOppg} later summarized the concept of HPFC in snake robots with special emphasis on the method of West and Asada \cite{west1985method}.

\section{Scope of the project}
The goal of this project is to establish a suitable and simplified 2-dimensional model of the idealized robot and its environment. A simulator is then to be developed from scratch based on this model and the concepts of OAL and HPFC. The purpose of the simulator is from the project task interpreted to be a platform for showing the aforementioned concepts on snake robots, rather than data generation for physical purposes.
Seeing as HOAL still is a fresh area of research and has little solid theoretical backdrop, this work goes in the direction of a proof-of-concept and experiments were performed to verify the method.

\subsection{Simplifications}
Due to the strict time constraints of the project and somewhat vague outlines, simplifications have been made to provide for a specific end goal. The simulator is restricted to handle bounded and well-defined scenarios with slow dynamics. Furthermore, the calculation of interaction forces in the simulator is neglected in favor of an easier discrete projection method that only takes joint velocities into consideration. This geometrical approximation is explained in further detail in Chapter \ref{ch:simulator} and discussed in Chapter \ref{ch:discussion}. The simplifications have led to some deviations from the true dynamical behavior of the robot when it is in contact with obstacles. As the velocity is changed without regard to energy conservation, this is most prominent in cases where the velocity is increased rather than decreased.

All assumptions regarding the employed model can be found in Chapter \ref{ch:model_specs}.

\subsection{Contributions}
A lot of the contributions connected to the limited research environment of HOAL have been clarification of the associated formulations and theoretical aspects.
Additionally, specific contributions are 

\begin{itemize}
    \item Study and analysis of HPFC, both with and without a dynamics consideration.
    \item Develpoment of a mathematical model of the snake robot and its environment           related to the given problem formulation.
    \item MATLAB simulator for demonstration of simple and germane HOAL-scenarios.
        It should be noted that this simulator is programmed from scratch and made in the context of an experimental study.
        On the basis of the generalized structure of the simulator program, several of the related modules can be applied to future implementations.
    \item Testing of the simulator performance and an evaluation of the associated             limitations.
    \item Evaluation of the OAL concept and discussion of remaining challenges, as well     as proposals for improvement.
\end{itemize}

\section{Report structure}
It is assumed that the reader is familiar with basic robotics and control theory. Please see \cite{lynch2017modern}, \cite{lynch2017modernCompTorque}, \cite{waldron2016kinematics}, \cite{liljeback2012snake} for a more thorough explanation of the topics.

The report first introduces the specifics around the mathematical model used for the snake robot, its environment and task in Chapter \ref{ch:model_specs}. Further, a review of the required background theory for understanding the employed dynamical model and control of the snake robot is presented in Chapter \ref{chapter:theory}. Chapter \ref{ch:simulator}, is focused on the simulator and method used for achieving OAL. Significant parts of the simulator are explained in detail and a guide for usage is provided. Chapter \ref{ch:experiments} presents some experiments performed with the simulator, which are discussed in Chapter \ref{ch:discussion} with focus on limitations and possibilities related to both the simulator and OAL in general. Lastly, Chapter \ref{outlook} briefly proposes some ideas for future work based on the challenges encountered in this project.